\section{CIP Under Uncertainty}

This appendix describes a theoretical contribution that is separate from the rest of this thesis.  Covered interest parity is one of the basic concepts of international finance.  Its description --- that investing in a domestic or a foreign currency should give the same return --- is intuitive.  In a deterministic environment, it is a no-arbitrage condition.  However, in an environment with randomness, the usual relationship does not have to hold by arbitrage.

\subsection{The Model}

My model is dervied from the Binomial Options Pricing Model\cite{Cox1979}.  However, instead of one stock price that can go up or down in the future, there are two interest rates (USD, EUR) that can go up or down.  

The initial state will be labeled $*$.  The state will transition to one of four second states $Uu$, $Ud$, $Du$, and $Dd$.  $U$ and $D$ indicate that the interest rate for USD has gone up or down, while $u$ and $d$ represent the same for EUR.  For simplicity, I assume the second states are equally probable.  

For claritly, time is indexed separately from the state.  Time goes from $t=0$ to $t=2$.  In general, the initial state resolves values in the time period $t=0$ to $t=1$ and the second states resolve values in $t=1$ to $t=2$. 

Interest rates are calculated between two times.  $r_{c,t_1,t_2,s}$ is the interest rate for currency $c$ from time $t_1$ to time $t_2$ as calculated in state $s$.  The interests rates for USD and EUR from $t=0$ to $t=1$ are given for the initial state.  These are $r_{USD,0,1,*}$ and $r_{EUR,0,1,*}$.  The interest rates for USD and EUR from $t=1$ to $t=2$ are given in the second states.  Thus, in state $Ud$, the value is $r_{USD,1,2,Ud}$.  

The interest rates for USD and EUR from $t=1$ to $t=2$ are \emph{not} given for the initial state, but these values could be calculated given traders behavior, the value in each of the second states, and the probability of transitioning to each second state.  This will be examined in the next section.

$S_{t,s}$ is the spot rate at time $t$ in state $s$.  The spot rate is given for time $t=2$ in all second states.  Thus, in state $Du$, the rate is written $S_{2,Du}$.  The spot rate is not given in $t=0$ nor $t=1$ for any state.


\subsection{Analysis}

Given the data known in the second states, the values for the spot price, forward price, and interest rates can be calculated for the initial state.  When we consider the values for the time span $t=0$ to $t=2$ in the initial state, only under certain conditions will those values obey the usual formula for CIP basis.  In general, they will not.

The value for the spot rate at time $t=1$ is calculated in each of the second states.  For any second state $s$, 
\begin{equation}
  S_{1,s} = \frac{1+r_{USD,1,2,s}}{1 + r_{EUR,1,2,s}} S_{2,s}
\end{equation}

\noindent This is just an application of the usual CIP formula, which can be applied here because the situation is completely deterministic.  

The value for the spot rate at time $t=1$ is cacluated for the initial state.  $S_{1,*} = \E[S_{1,s}]$ where $s$ is restricted to second states.  The result is:
\begin{equation}
  S_{1,*} = . 25\cdot S_{1,Uu} + .25\cdot S_{1,Ud} + .25\cdot S_{1,Du} + .25\cdot S_{1,Dd} 
\end{equation}

The spot rate's value at $t=0$ in the initial state is an application of the usual CIP formula, since the situation is deterministic inside the initial state.
\begin{equation}
  S_{0,*} = \frac{1+r_{USD,0,1,s}}{1 + r_{EUR,0,1,s}}(. 25\cdot S_{1,Uu} + .25\cdot S_{1,Ud} + .25\cdot S_{1,Du} + .25\cdot S_{1,Dd})
\end{equation}

$F_{t_1,t_2,s}$ is the price of a forward bought at $t_1$ with delivery date $t_2$, as calculated in state $s$.  We are only concerned with forwards that deliver at time $t=2$.  Inside a second state $s$, we know $F{1,2,s} = S_{2,s}$ because we know the spot price at time $t=2$ and the forward price, in a deterministic environment, should be equal to it.

The price of a forward at $t=1$ in the initial state are set by a risk-neutral measure.  A marketmaker buying and selling forwards treats profits and loses in any future state equally, and thereby sets the price at the expected value in each possible state.  $F_{1,2,*} = \E[F_{1,2,s}]$ where $s$ is restricted to second states.

The price of a forward at $t=0$ in the initial state is equal to the price at $t=1$ in the initial state, because it is a deterministic environment.  Thus, $F_{0,2,*} = F_{1,2,*} = \E[F_{1,2,s}] = \E[S_{2,s}]$ for $s$ restricted to second states.  Expanded, we get:
\begin{equation}
  F_{0,2,*} =. 25\cdot S_{2,Uu} + .25\cdot S_{2,Ud} + .25\cdot S_{2,Du} + .25\cdot S_{2,Dd}
\end{equation}

\todo{FIGURE OUT IF THE ``risk-neutral'' COMMENT IN THE NEXT LINE IS B.S. OR NOT}

Lastly, we need interest rates from $t=0$ to $t=2$.  The interest rate for the second half, from $t=1$ to $t=2$, is given in each second state $s$ as $r_{c,1,2,s}$.  If interest profits follow a risk-neutral measure, the interest rate for the second half, as seen in the initial state, is the expected value of each possible state.  $r_{1,2,*} = \E[r_{1,2,s}]$ for $s$ restricted to second states.  The interest rate for the whole $t=0$ to $t=2$ must then obey:
\begin{equation}
  (1 + r_{c,0,2,*})^2 = (1 + r_{c,0,1,*})(1 + .25\cdot r_{c,1,2,Uu} + .25\cdot r_{c,1,2,Ud} + .25\cdot r_{c,1,2,Du} + .25\cdot r_{c,1,2,Dd})
\end{equation}

\noindent for a currency $c$.

Now that we have the spot price, forward price and interest rates for the period $t=0$ to $t=2$ in the initial state, we can plug them into the traditional CIP basis formula.  The CIP basis formula for two time periods is:
\begin{equation}
  S_{0,*} = \frac{(1 + r_{USD,0,2,*})^2}{(1 + r_{EUR,0,2,*})^2}F_{0,2,*}
\end{equation}

Substituting, we get:
\begin{equation}
  \frac{1+r_{USD,0,1,s}}{1 + r_{EUR,0,1,s}}\E[\frac{1+r_{USD,1,2,s}}{1 + r_{EUR,1,2,s}} S_{2,s}] = \frac{(1 + r_{USD,0,1,*})(1 +\E[r_{USD,1,2,s}])}{(1 + r_{EUR,0,1,*})(1 +\E[r_{EUR,1,2,s}])} \E[S_{2,s}]
\end{equation}

Reducing, we get:
\begin{equation}
  \E[\frac{1+r_{USD,1,2,s}}{1 + r_{EUR,1,2,s}} S_{2,s}] = \frac{(1 +\E[r_{USD,1,2,s}])}{(1 +\E[r_{EUR,1,2,s}])} \E[S_{2,s}]
\end{equation}






\subsection{Discrete Model}

\todo{Find lots of quotes about CIP holding always, etc.}

My model is dervied from the Binomial Options Pricing Model\cite{Cox1979}.  However, instead of one stock price that can go up or down in the future, there are two interest rates (USD, EUR) that can go up or down.  

\todo{FIX THIS DIAGRAM.  There are 4 states at $t=1$}

\begin{figure}[H]  % h = here
\includegraphics[width=\textwidth]{images/SimpleStateDiagram}
\caption{States in the model.  Interest rates for the period $t=1$ to $t=2$ can go up or down for USD and/or  EUR.}
\label{SimpleStateDiagram}
\end{figure}

Given those interest rates and the FX spot prices at $t=2$, the model can determine the FX spot price, FX forward price, and long-term (covering both period) interest rates at $t=0$.  I'll then plug those into the CIP formula to see what other conditions must hold in order for CIP to hold.  

The model has two currencies USD and EUR, with time running from $t=0$ to $t=2$.  There are two periods: one running from $t=0$ to $t=1$ and the other from $t=1$ to $t=2$.  At $t=0$, short-term interest rates are known for the first period: $r_{USD,1}$ and $r_{EUR,1}$.  However, at $t=0$, the short-term interest rates for the second period are yet not determined.  At $t=1$, the rate for USD has a 50\%/50\% chance of becoming $r_{USD,U}$ or $r_{USD,D}$.  It is similar for EUR, but I'll index them using lower-case letters, so $r_{EUR,u}$ and $r_{EUR,d}$.  For simplicity of description, I will use $t=1-$ to refer to the state at $t=1$ with future rates unknown and $t=1+$ to refer to the state with future rates known.

\begin{figure}[H]  % h = here
\includegraphics[width=\textwidth]{images/StateDiagramExpandedT1}
\caption{For clarity, it is sometimes useful to talk about the state at $t=1$ before uncertainty is resolve and after.  Before is termed $t=1-$ and after is $t=1+$}
\label{StateDiagramExpandedT1}
\end{figure}

The other exogenous values in the model are the FX spot prices at $t=2$.  These are called $FX_{2,Uu}$, $FX_{2,Ud}$, $FX_{2,Du}$ and $FX_{2,Dd}$.  Knowing the values at $t=2$, we can now compute backwards and to get the FX prices at $t=0$.

There are 4 FX spot prices at time $t=1+$ because the interest rates for the second period have already been determined.  This is a determinisic setting, so CIP holds for this one time period.  $1+r_{USD,U} = FX_{1+,Uu}(1+r_{EUR,u})/FX_{2,Uu}$  Solving for the FX spot price yields:  $FX_{1+,Uu} = (1+r_{USD,U})/(1+r_{EUR,u})FX_{2,Uu}$.  

At time $t=1-$, the second period rates have not been determined.  There is 25\% of each happening, so the FX spot price is:
\[ FX_{1-} = .25 \frac{1 + r_{USD,U}}{1 + r_{EUR,u}}FX_{2,Uu} + 
                                                            .25 \frac{1+r_{USD,U}}{1+r_{EUR,d}}FX_{2,Ud} + 
                                                            .25 \frac{1+r_{USD,D}}{1+r_{EUR,u}}FX_{2,Du} + 
                                                            .25 \frac{1+r_{USD,D}}{1+r_{EUR,d}}FX_{2,Dd} \]

At time $t=0$, the movement to $t=1-$ is deterministic, so we can applying the CIP formula again.  The result is:
\[ FX_{0} = \frac{1+r_{USD,1}}{1+r_{EUR,1}}( .25 \frac{1+r_{USD,U}}{1+r_{EUR,u}}FX_{2,Uu} + 
                                                            .25 \frac{1+r_{USD,U}}{1+r_{EUR,d}}FX_{2,Ud} + 
                                                            .25 \frac{1+r_{USD,D}}{1+r_{EUR,u}}FX_{2,Du} + 
                                                            .25 \frac{1+r_{USD,D}}{1+r_{EUR,d}}FX_{2,Dd}) \]

In the model, we only care about FX forwards that expire at $t=2$. \footnote{For completeness, the FX forwards that expires at $t=1$ can only be bought at $t=0$.  In that case, the forward price is the average of FX spot prices at $t=1+$.  This gives us:
\[ FWD_{0,1} = .25 \frac{1 + r_{USD,U}}{1 + r_{EUR,u}}FX_{2,Uu} + 
                                                            .25 \frac{1+r_{USD,U}}{1+r_{EUR,d}}FX_{2,Ud} + 
                                                            .25 \frac{1+r_{USD,D}}{1+r_{EUR,u}}FX_{2,Du} + 
                                                            .25 \frac{1+r_{USD,D}}{1+r_{EUR,d}}FX_{2,Dd} \]}  The FX forwards bought at $t=1+$ are the same as FX spot prices at $t=2$, because we know what those prices will be at $t=2$.  The FX forward prices for $t=1-$ are set by risk neutral measure, which is the mean of the prices at $t=2$.  \todo{Write that cleaner.}
\[ FWD_{1-} = .25 FX_{2,Uu} +.25 FX_{2,Ud} +.25 FX_{2,Du} +.25 FX_{2,Dd} \]  
The price at $t=0$ is the same.
\[ FWD_{0} = .25 FX_{2,Uu} +.25 FX_{2,Ud} +.25 FX_{2,Du} +.25 FX_{2,Dd} \]  

Lastly, to compute CIP from $t=0$ to $t=2$, we need interest rates for those time periods.  
\[(1 + r_{USD,long})^2 = (1+r_{USD,1})(.5(1+r_{USD,U}) + .5(1+r_{USD,D})) \]
\[(1 + r_{EUR,long})^2 = (1+r_{EUR,1})(.5(1+r_{EUR,u}) + .5(1+r_{EUR,d})) \]

CIP says $(1 + r_{USD,long})^2 = FX_{0} (1 + r_{EUR,long})^2 / FWD_{0}$.   Substituting and simplifying gets us:

% \[ (1+r_{USD,1})(.5(1+r_{USD,U}) + .5(1+r_{USD,D})) = (\frac{1+r_{USD,1}}{1+r_{EUR,1}}( .25 \frac{1+r_{USD,U}}{1+r_{EUR,u}}FX_{2,Uu} + 
%                                                             .25 \frac{1+r_{USD,U}}{1+r_{EUR,d}}FX_{2,Ud} + 
%                                                             .25 \frac{1+r_{USD,D}}{1+r_{EUR,u}}FX_{2,Du} + 
%                                                             .25 \frac{1+r_{USD,D}}{1+r_{EUR,d}}FX_{2,Dd})) (1+r_{EUR,1})(.5(1+r_{EUR,u}) + .5(1+r_{EUR,d})) / (.25 FX_{2,Uu} +.25 FX_{2,Ud} +.25 FX_{2,Du} +.25 FX_{2,Dd}) \]


\[ \frac{(1+r_{USD,U}) + (1+r_{USD,D})}{(1+r_{EUR,u}) + (1+r_{EUR,d})} = \frac{ \frac{1+r_{USD,U}}{1+r_{EUR,u}}FX_{2,Uu} + 
                                                            \frac{1+r_{USD,U}}{1+r_{EUR,d}}FX_{2,Ud} + 
                                                            \frac{1+r_{USD,D}}{1+r_{EUR,u}}FX_{2,Du} + 
                                                            \frac{1+r_{USD,D}}{1+r_{EUR,d}}FX_{2,Dd}}{FX_{2,Uu} +FX_{2,Ud} +FX_{2,Du} + FX_{2,Dd}} \]

Reorganizing:

\[ FX_{2,Uu} +FX_{2,Ud} +FX_{2,Du} + FX_{2,Dd} = 
\frac {(1+r_{EUR,u}) + (1+r_{EUR,d})}{(1+r_{USD,U}) + (1+r_{USD,D})} (\frac{1+r_{USD,U}}{1+r_{EUR,u}}FX_{2,Uu} + 
                                                            \frac{1+r_{USD,U}}{1+r_{EUR,d}}FX_{2,Ud} + 
                                                            \frac{1+r_{USD,D}}{1+r_{EUR,u}}FX_{2,Du} + 
                                                            \frac{1+r_{USD,D}}{1+r_{EUR,d}}FX_{2,Dd}) \]

If this equation is going to hold for any FX spot prices, then we separate the equation into 4 equations, one for each $FX_{2,?}$.

\begin{equation}
  \begin{aligned}
 1 & = \frac {(1+r_{EUR,u}) + (1+r_{EUR,d})}{(1+r_{USD,U}) + (1+r_{USD,D})}\frac{1+r_{USD,U}}{1+r_{EUR,u}}\\
 1 & = \frac {(1+r_{EUR,u}) + (1+r_{EUR,d})}{(1+r_{USD,U}) + (1+r_{USD,D})}\frac{1+r_{USD,U}}{1+r_{EUR,d}} \\
 1 & = \frac {(1+r_{EUR,u}) + (1+r_{EUR,d})}{(1+r_{USD,U}) + (1+r_{USD,D})}\frac{1+r_{USD,D}}{1+r_{EUR,u}} \\
 1 & = \frac {(1+r_{EUR,u}) + (1+r_{EUR,d})}{(1+r_{USD,U}) + (1+r_{USD,D})}\frac{1+r_{USD,D}}{1+r_{EUR,d}}
  \end{aligned}
\end{equation}

Combining the first two equation, 
\[ \frac {(1+r_{EUR,u}) + (1+r_{EUR,d})}{(1+r_{USD,U}) + (1+r_{USD,D})}\frac{1+r_{USD,U}}{1+r_{EUR,d}} = \frac {(1+r_{EUR,u}) + (1+r_{EUR,d})}{(1+r_{USD,U}) + (1+r_{USD,D})}\frac{1+r_{USD,U}}{1+r_{EUR,u}} \]

Which reduces to $r_{EUR,d} =r_{EUR,u}$.   Similarly, combining the second and fourth equation yields $r_{USD,U} = r_{USD,D}$.  Together, those equations say there is no randomness at time $t=2$.  So, the only way for CIP to hold with any value for time $t=2$ FX spot prices, is for there to be no randomness.  The alternative is to say is randomness and CIP holds, but there are constraints on the exogenous values of $t=2$ FX spot prices.  And exogenous values should not have constraints.  

Thus, with any randomness, CIP does not have to hold.

